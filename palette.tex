\documentclass[a4]{article}
\usepackage{xcolor}
\usepackage{xcolor-material}
\usepackage{tikz}

%\makeatletter
%\pgfkeys{/palette/.is family, /palette,
%      width/.estore in = \XCM@pltWidth,
%      height/.estore in = \XCM@pltHeight,
%      shape/.estore in = \XCM@pltShape,
%      title text color/.estore in = \XCM@pltTTextColor,
%      first text color/.estore in = \XCM@pltFTextColor,
%      second text color/.estore in = \XCM@pltSTextColor,
%      colorbox variant font/.store in = \XCM@pltVFont,
%      colorbox value font/.store in = \XCM@pltNFont,
%      title font/.store in = \XCM@pltTFont,
%      colorbox sep/.estore in = \XCM@pltColorboxSep,
%      first colorbox sep/.estore in = \XCM@pltFirstCSep,
%      title colorbox height/.estore in = \XCM@pltTitleHeight,
%      color variant/.estore in = \XCM@pltVariant,
%      add variant/.estore in = \XCM@pltAddVariant,
%      percent char/.estore in = \XCM@pltPercentSep,
%      primary variant/.estore in = \XCM@pltPrimary,
%      change at/.estore in = \XCM@pltChange,
%      add change at/.estore in = \XCM@pltAddChange,
%      color model/.estore in = \XCM@ColorModel,
%      default/.style = {width = .45\textwidth,
%            height = 2\baselineskip,
%            shape = rectangle,
%            title text color = black,
%            first text color = black,
%            second text color = white,
%            colorbox sep = 0pt,
%            first colorbox sep = .2\baselineskip,
%            title colorbox height = 6\baselineskip,
%            colorbox variant font = \sffamily,
%            colorbox value font = \ttfamily,
%            title font = \sffamily\bfseries,
%            color variant = {10,20,...,100},
%            add variant = {},
%            percent char = !,
%            primary variant = 100,
%            change at = 1000,
%            add change at = 1000,
%            color model = HTML
%      }
%}
%\usetikzlibrary{positioning}
%\newcommand*{\colorpalette}[2][]{%
%      \pgfkeys{/palette, default, #1}%
%      \def\XCM@colorname{#2}%
%      \begin{tikzpicture}
%            \tikzset{colorbox/.style={\XCM@pltShape, minimum width=\XCM@pltWidth, minimum height=\XCM@pltHeight, node distance=\XCM@pltColorboxSep, outer sep=0pt}}
%            \tikzset{title/.style = {font=\XCM@pltTFont, color=\XCM@pltTTextColor, inner sep=1em}}
%            \tikzset{variant/.style = {font=\XCM@pltVFont, inner sep=1em}}
%            \tikzset{value/.style = {font=\XCM@pltNFont, inner sep=1em}}
%            
%            \node[colorbox, fill=\XCM@colorname\XCM@pltPercentSep\XCM@pltPrimary, minimum height=\XCM@pltTitleHeight, alias=last] {};
%            \node[title, anchor=north west] at (last.north west) {\XCM@colorname};
%            \node[title, anchor=south west] at (last.south west) {\XCM@pltPrimary};
%            \node[title, anchor=south east, align=right] at (last.south east) {\texttt{\printcolorvalue[\XCM@ColorModel]{\XCM@colorname\XCM@pltPercentSep\XCM@pltPrimary}}};
%            
%            \def\auxcolor{\XCM@pltFTextColor}%
%            \def\first{1}%
%            \foreach \variant [count=\i] in \XCM@pltVariant {
%                  \ifx\i\first
%                        \node[node distance=\XCM@pltFirstCSep, inner sep=0pt, below=of last, alias=last] {};
%                  \else\fi
%                  \ifx\variant\XCM@pltChange
%                        \gdef\auxcolor{\XCM@pltSTextColor}%
%                  \else\fi
%                  \node[colorbox, fill=\XCM@colorname\XCM@pltPercentSep\variant, below=of last, alias=last] {};
%                  \node[variant, anchor=west, color=\auxcolor] at (last.west) {\variant};
%                  \node[value, anchor=east, color=\auxcolor, align=right] at (last.east) {\printcolorvalue[\XCM@ColorModel]{\XCM@colorname\XCM@pltPercentSep\variant}};
%            }
%            
%            \def\auxcolor{\XCM@pltFTextColor}%
%            \foreach \variant [count=\i] in \XCM@pltAddVariant {
%                  \ifx\i\first
%                        \node[node distance=\XCM@pltFirstCSep, inner sep=0pt, below=of last, alias=last] {};%
%                  \else\fi 
%                  \ifx\variant\XCM@pltAddChange
%                        \gdef\auxcolor{\XCM@pltSTextColor}%
%                  \else\fi
%                  \node[colorbox, fill=\XCM@colorname\XCM@pltPercentSep\variant, below=of last, alias=last] {};
%                  \node[variant, anchor=west, color=\auxcolor] at (last.west) {\variant};
%                  \node[value, anchor=east, color=\auxcolor, align=right] at (last.east) {\printcolorvalue[\XCM@ColorModel]{\XCM@colorname\XCM@pltPercentSep\variant}};
%            }
%      \end{tikzpicture}
%}
%\makeatother
\newlength{\pltheight}
\setlength{\pltheight}{\baselineskip}
\newlength{\pltwidth}
\setlength{\pltwidth}{.4\textwidth}
\newif\ifaddvariant
\addvariantfalse
%\newcommand*{\colorpalette}[2][]{%
%  \def\materialcolor{#2}
%  \parbox[t]{\pltwidth}{%
%    \textcolor{\materialcolor 50}{\rule{\pltwidth}{\pltheight}}\\
%    \textcolor{\materialcolor 100}{\rule{\pltwidth}{\pltheight}}\\
%    \textcolor{\materialcolor 200}{\rule{\pltwidth}{\pltheight}}\\
%    \textcolor{\materialcolor 300}{\rule{\pltwidth}{\pltheight}}\\
%    \textcolor{\materialcolor 400}{\rule{\pltwidth}{\pltheight}}\\
%    \textcolor{\materialcolor 500}{\rule{\pltwidth}{\pltheight}}\\
%    \textcolor{\materialcolor 600}{\rule{\pltwidth}{\pltheight}}\\
%    \textcolor{\materialcolor 700}{\rule{\pltwidth}{\pltheight}}\\
%    \textcolor{\materialcolor 800}{\rule{\pltwidth}{\pltheight}}\\
%    \textcolor{\materialcolor 900}{\rule{\pltwidth}{\pltheight}}
%    
%    \ifaddvariant
%        \vspace{.1\pltheight}
%        \textcolor{\materialcolor A100}{\rule{\pltwidth}{\pltheight}}\\
%        \textcolor{\materialcolor A200}{\rule{\pltwidth}{\pltheight}}\\
%        \textcolor{\materialcolor A400}{\rule{\pltwidth}{\pltheight}}\\
%        \textcolor{\materialcolor A700}{\rule{\pltwidth}{\pltheight}}
%     \else\fi
%  }    
%}
\begin{document}
	\colorbox{MaterialGreen}{\parbox[b][10\baselineskip]{8em}{\hspace{1em}\ttfamily\color{white}\vspace*{1em} MaterialGreen\vfill\printcolorvalue{MaterialGreen}\vspace*{1em}}}
	\colorbox{MaterialGreen}{\parbox[b][10\baselineskip]{8em}{\hspace{1em}\ttfamily\color{white}\printcolorvalue{MaterialGreen600}\vspace*{1em}}}
%      \colorpalette[color variant={50,100,200,...,900}, percent char={}, primary variant=500, add variant={A100,A200,A400,A700}, title text color=white, change at=400, add change at=A200]{MaterialRed}
%
%      \colorpalette[color variant={50,100,200,...,900}, percent char={}, primary variant=500, add variant={A100,A200,A400,A700}, title text color=white, change at=300, add change at=A200]{MaterialPink}
%      
%      \colorpalette[color variant={50,100,200,...,900}, percent char={}, primary variant=500, add variant={A100,A200,A400,A700}, title text color=white, change at=300, add change at=A200]{MaterialPurple}
%      \colorpalette[color variant={50,100,200,...,900}, percent char={}, primary variant=500, add variant={A100,A200,A400,A700}, title text color=white, change at=300, add change at=A200]{MaterialDeepPurple}
%            
%      \colorpalette[color variant={50,100,200,...,900}, percent char={}, primary variant=500, add variant={A100,A200,A400,A700}, title text color=white, change at=500, add change at=A200]{MaterialBlue}
%      \colorpalette[color variant={50,100,200,...,900}, percent char={}, primary variant=500, add variant={A100,A200,A400,A700}, title text color=white, change at=300, add change at=A200]{MaterialIndigo}
%      
%      \colorpalette[color variant={50,100,200,...,900}, percent char={}, primary variant=500, add variant={A100,A200,A400,A700}, change at=600, add change at=A700]{MaterialLightBlue}
%      \colorpalette[color variant={50,100,200,...,900}, percent char={}, primary variant=500, add variant={A100,A200,A400,A700}, change at=700]{MaterialCyan}
%      
%      \colorpalette[color variant={50,100,200,...,900}, percent char={}, primary variant=500, add variant={A100,A200,A400,A700}, title text color=white, change at=500]{MaterialTeal}
%      \colorpalette[color variant={50,100,200,...,900}, percent char={}, primary variant=500, add variant={A100,A200,A400,A700}, change at=600]{MaterialGreen}
%      
%      \colorpalette[color variant={50,100,200,...,900}, percent char={}, primary variant=500, add variant={A100,A200,A400,A700}, change at=700]{MaterialLightGreen}
%      \colorpalette[color variant={50,100,200,...,900}, percent char={}, primary variant=500, add variant={A100,A200,A400,A700}, change at=900]{MaterialLime}
%      
%      \colorpalette[color variant={50,100,200,...,900}, percent char={}, primary variant=500, add variant={A100,A200,A400,A700}]{MaterialYellow}
%      \colorpalette[color variant={50,100,200,...,900}, percent char={}, primary variant=500, add variant={A100,A200,A400,A700}]{MaterialAmber}
%      
%      \colorpalette[color variant={50,100,200,...,900}, percent char={}, primary variant=500, add variant={A100,A200,A400,A700}, change at=800]{MaterialOrange}
%      \colorpalette[color variant={50,100,200,...,900}, percent char={}, primary variant=500, add variant={A100,A200,A400,A700}, title text color=white, change at=500, add change at=400]{MaterialDeepOrange}
%      
%      \colorpalette[color variant={50,100,200,...,900}, percent char={}, primary variant=500, title text color=white, change at=300]{MaterialBrown}
%      \colorpalette[color variant={50,100,200,...,900}, percent char={}, primary variant=500, change at=600]{MaterialGrey}
%      
%      \setlength{\pltheight}{2\baselineskip}
%      \colorpalette[color variant={50,100,200,...,900}, percent char={}, primary variant=500, title text color=white, change at=400]{MaterialBlueGrey}
      
%      \colorpalette[color variant={red, blue, yellow, gray, black, purple}, percent char={}, primary variant=orange, add variant={red!20, red!30,blue!20, blue!40}]{}
      
      %\colorpalette[height=1cm, color model=RGB]{yellow}
      
      %\colorpalette[height=1cm, color model=RGB, colorbox sep=.5cm]{yellow}
      
      %\colorpalette[colorbox sep=.5cm]{brown}
\end{document}